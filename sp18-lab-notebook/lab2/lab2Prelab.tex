\section{Static Methods}

\subsection{Methods}
The algorithms we looked at last week mostly share something in common: they work over some input data and produce some output data. Wouldn't it be nice if we could package these algorithms together and run them without having to type in each individual command? 

Methods in Java are a way to do just this. When you define a method (which we'll do in the next unit), you specify what kind of input and output the method will have, and then you add a series of statements that process that data. 

There are many of these methods already present in the Java API (Application Programming Interface -- a large collection of Java code for various common tasks that ships for free with our REPL). We can run these methods by using a method call expression. 

\subsection{Static Methods}

There are two types of methods in Java -- static methods and instance methods. We'll work with both this week. A static method resides in what is called a class in Java. When we want to call a static method, we first provide the name of the class, then a dot, then the name of the method we want to call. Following that, we open up a set of parenthesis to hold any arguments we want to pass in to the method (this is the input data the method works over). When the method is done running, it will return the output data to the REPL. 

\begin{exa}
Inside the Math class, there is a method called sqrt (short for square root). In the REPL, we can form a method call expression that determines the square root of 4 as follows:

\begin{code}
Math.sqrt(4.0)
\end{code}

This expression yields the value 2.0.

\end{exa}

Check the documentation for the Math class for a complete list of all the static methods you can call. 

\subsection{More Complex Expressions}

Last week we used operators to combine expressions in to more complex expressions. We can also use method calls to create more complex expressions. Method call expressions can be nested inside of other method call expressions and operators can join method call expressions (just as they do with any other expression). 

Some methods require more than one argument, such as the Math.pow method, which takes two arguments (a base value and an exponent value). The method returns the base value raised to the power of the exponent value. Let's see how we can use this method to build a more complex expression for a familiar mathematical formula. 

\begin{exa}
The example below creates four double variables named x1, y1, x2, and y2 representing the x and y coordinates of two different points on the Cartesian plane. We can compute the distance between these points with the Pythagorean theorem, represented by the complex Java expression below:

\begin{code}
double x1 = 1.0;
double x2 = 4.0;
double y1 = 3.0;
double y2 = 7.0;
Math.sqrt(Math.pow(x1 - x2, 2) + Math.pow(y1 - y2, 2))
\end{code}
\end{exa}

\section{Instance Methods}

The other type of method in Java is an instance method. Instead of being called from the context of a class, an instance method is called from the context of an object instance. Of course, before this will make sense, we first need to understand what an object is!

\subsection{Object Types}

So far the only data types we've seen in Java are what are called primitive types (types like int and double). There are only a limited number of primitive types in Java (we've already seen nearly half of them). There is another category of more complex data types called object types. There are an unlimited number of these types since you can develop new object types yourself (and indeed you will if you take the next course in this sequence). 

Object types are at the heart of the paradigm known as Object Oriented programming. Most classes in Java are used as definitions of object types. Individual instances of these classes are the values of these types. 

Object types can represent much more complex and abstract concepts than the primitive types we've seen. For example, consider that we needed to use two different variables in the previous example to represent a point on the Cartesian plane. In the Java API, there is a class that represents the more abstract concept of a two dimensional point. This class is called, not surprisingly, Point. We'll review our Pythagorean Theorem example using this data type, but first we'll need to learn a new type of Java statement that helps organize the massive amount of classes (literally thousands) in the Java API. 

\subsection{Java Packages and Import Statements}

Because there are so many classes in the Java API, and also because many Java programming projects define potentially hundreds more, most classes are organized in to what are called packages in Java. To access a class organized in to a package, we must do one of two things:

\begin{itemize}
\item \textbf{Use the Full Name of the Class} - The full name of a class includes its package name. Package names are a series of Java identifiers separated by periods. The Point class resides in the java.awt package, so we can use this class by referring to it with its full name: java.awt.Point
\item \textbf{Import the Class} - Using the prior method can be a pain because the full names are quite long. More often we use what are called import statements to let the REPL know that when we say Point, we are referring to the java.awt.Point class. An import statement consists of the keyword import followed by the full name of the class we want to refer to with shorthand. For example, to import that java.awt.Point class, we would use the following statement:

\begin{code}
import java.awt.Point;
\end{code}

Alternatively, we can import every class from the java.awt package all at once using the following statement (which uses a wildcard):

\begin{code}
import java.awt.*;
\end{code}

However, if we do this with more than one package, we run the risk of importing two classes with the same name, causing a conflict! For example, there is a class called List in java.awt and also a class called list in java.util; two commonly used packages!
\end{itemize}

\subsection{Creating Object Instances}

When we want to create a value of a primitive type, we can simply use a literal expression. However, if we want to create an object value (called an object instance), we need to call a constructor -- a special kind of method that creates object instances. 

To call a constructor, you first must use the new keyword. Following that is the name of the class you want to create an instance of, then a set of parentheses containing any arguments you want to pass in to the constructor method. If there are no errors, the constructor returns a new instance of the class you specified. 

\begin{exa}
The constructor for the point class requires two arguments -- the x and y coordinates for the point. Assume the Point class has already been imported. The following statement will create a new variable named p of type Point, create an instance of the Point class representing the origin, and assign that value to the variable:

\begin{code}
Point p = new Point(0, 0);
\end{code} 
\end{exa}

\subsection{Calling Instance Methods}

Once we have an object instance, we work with it by calling instance methods. As I mentioned before, instance methods are called from the context of an object instance. The methods work over the state of the object. To call an instance method, we start with an expression that yields an object-type value (this is typically just a variable expression where the variable holds the object value we're interested in). Next we include a dot, then the name of the instance method, and finally a set of parentheses containing any arguments we want to pass in to the method. 

\begin{exa}
The Point object we created in the last example represents the location (0, 0) on the Cartesian plane, but we can move it to a new position by calling the translate instance method. The translate method is somewhat special in that it doesn't return a value -- it simply alters the internal state of the point object we call it from. When we call such a method, it is written as a statement instead of an expression (that is, we include a semicolon at the end). The following statement moves the point p to the position (3, 4):

\begin{code}
p.translate(3, 4);
\end{code}

The Point class also contains some simple instance methods that report back a portion of the object's state, such as one of the coordinates. These kinds of instance methods are commonly called getters. For example, calling the getX() method will yield 3.0:

\begin{code}
p.getX()
\end{code}

Note that getX doesn't require any arguments, so we didn't include any in the method call, but we still needed to include the parentheses (even if they're empty)!

\end{exa}

\begin{exa}
The following is a short series of statements that recreates the previous Pythagorean theorem example using two Point variable instead of four double variables:

\begin{code}
import java.awt.Point;
Point p1 = new Point(1, 3);
Point p2 = new Point(4, 7);
Math.sqrt(Math.pow(p1.getX() - p2.getX(), 2.0) + Math.pow(p1.getY() - p2.getY(), 2.0))
\end{code}

\end{exa}


\section{Casting}

Note that object type variables are incompatible with primitive values. For instance, assigning the value 5 to p1 from the prior section will result in an error. However, some primitive values are compatible with each other or can be made compatible through a process called casting. For instance, the following is legal:

\begin{code}
double x = 5;
\end{code}

Technically, 5 is an integer value expression but x is a double value variable. 5 is automatically converted to the double value 5.0 through a process generally called type coercion. The opposite situation doesn't work:

\begin{code}
int y = 5.0;
\end{code}

The concern is that the double value being assigned might not exactly represent an integer (for instance, what should 3.4 be converted in to? 3? 4?). For the conversion to take place here, we must make our intentions to convert the value explicit by including the name of the type we want to convert to in parentheses before the numeric expression:

\begin{code}
int y = (int)5.0;
\end{code}

Note that this cast will truncate the double value (that is, the expression (int)5.6 will yield the integer value 5). All numeric values can be cast to any numeric type. 
\newpage

\section{Preparing for Class}

Please answer the following questions before you arrive to class:

\begin{exer}

\begin{itemize}
\item What are the two broad categories of types in Java?

  \evalline
  
\item What do you call a method that is used to create a new object instance (calling such methods requires you to include the keyword \textbf{new} at the beginning of the method call expression)?

  \evalline
  
\item What term best describes a portion of Java code with a semicolon on the end (typically one step in a program)?

  \evalline
  
\item Give an example of any expression that evaluates to a numeric value

  \evalline
  
  
\end{itemize}

\end{exer}

When you're finished, scan this sheet (only this page is needed) and turn it in through the prelab \#2 assignment. 

