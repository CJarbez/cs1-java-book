\subsection{Part 1 -- Defining Methods}

We'll start this week with some basic practice with method definitions.

\begin{exer}
Enter the definitions for the max and abs functions from the prelab in to JShell. Then make calls to these methods with various argument values, confirming that the work.
\end{exer}

Now you'll define a new method, albeit with some code we've already written in a previous lab. 

\begin{exer}
Define a method named \textbf{isLeapYear}. This method should return a boolean value, and it should take one integer argument (representing a year). This method should return true if the year is a leap year, and false otherwise. Once your method is defined, test it with a variety of input to make sure it's working properly. 
\end{exer}



\begin{eval}

Enter each of the method definitions below in to the REPL. On the following line, record what happened in the REPL (you may summarize if it's complex). Then, on the line following that, explain in plain english why you got that result. Many of these expressions / statements will result in 
an error. If you're unsure how to explain it, refer back to the prelab and/or ask the instructor for help.

\begin{sevalenum}

\item
\begin{code}
int myMethod(double x) { 
  return x + 5;
}
\end{code}

\evallinetwo

\item 
\begin{code}
int myMethod(int x) { 
    x = 4;
}
\end{code}

\evallinetwo

\item 
\begin{code}
int myMethod(int x) { 
  return 5;
  x = 4;
}
\end{code}

\evallinetwo

\end{sevalenum}
\end{eval}


\initialbox


\subsection{Part 2 -- Unit Testing Methods}

Take a look at the \textbf{sleepIn} problem in Warmup-1 on CodingBat. 

\begin{exer}Define two boolean type variables in the REPL, one named \textbf{weekday} and one named \textbf{vacation}. Now develop a boolean expression that follows the rules specified in the \textbf{sleepIn} description. Test it out in the REPL. Write down the expression you used below:

\evalline

Now define a method in JShell named \textbf{sleepIn}. It should have the same method header you see on CodingBat (the \textit{public} keyword is not necessary, but doesn't harm anything -- you can leave it out). For the body of this method, use the expression you developed above in a single return statement. Copy and paste the tests on this page in to the REPL (copy everything before the arrows). The results should be the same as the boolean values following the arrows on this page. If they're not, edit your method to make sure it's correct. 

Next, copy and paste your return statement in to the method body on CodingBat and hit the ``Go'' button. To the right, you'll see some additional tests. All of these tests should pass (that is, CodingBat should indicate that your method is producing the expected output for each of these test cases). If it isn't, you'll need to again make some corrections to your method definition. Once you're finished, write your final boolean expression below (or just indicate that you were right the first time):

\evalline

\end{exer}

\initialbox


\subsection{Part 3 -- Methods with If Statements}

\begin{exer}
Enter the definition for the \textbf{justRight} method definition from the prelab, but altering it so that it no longer uses any if statements. You should be able to write a 1-line method body for this method. Provide that line below once you're confident you have the method working in JShell:

\evalline
\end{exer}

\begin{exer}
Now edit \textbf{isLeapYear} so that it doesn't use any boolean operators (that is, no and's or or's). You will need to use nested if statements to accomplish this. Once you have it tested and working, write down your definition below:

\evallinefive
\end{exer}


\begin{eval}
Enter each of the method definitions below in to the REPL. On the following line, record what happened in the REPL (you may summarize if it's complex). Then, on the line following that, explain in plain english why you got that result. Many of these expressions / statements will result in 
an error. If you're unsure how to explain it, refer back to the prelab and/or ask the instructor for help.

\begin{sevalenum}


\item 
\begin{code}
int myMethod(int x) { 
  if( 1 + 1 == 3 ) {
    return x + 5; 
  } 
}
\end{code}

\evallinetwo

\item 
\begin{code}
int myMethod(int x) { 
  if( 1 + 1 == 3) {
    x = 2;
  } else {
    return x + 9;
  }
}
\end{code}

\evallinetwo

\item 
\begin{code}
int myMethod() { 
  if( 1 + 1 == 2 ) {
    return 5; 
  } 
}
\end{code}

\evallinetwo


\end{sevalenum}
\end{eval}

\initialbox


\subsection{Part 4 -- More Method Writing Practice}

Now complete the following 3 problems from CodingBat. It may be helpful to define these in your REPL first so that you can test them interactively and get error messages for any mistakes you make, but you are welcome to define them in CodingBat first if you wish. Work closely with your partner as you discuss and develop your code. When you finish each one, write your solutions down in the space provided. 

\begin{exer}
\begin{enumerate}

\item \textit{Warmup-1} -- \textbf{MonkeyTrouble}
\evallinefive

\item \textit{Warmup-1} -- \textbf{makes10}
\evallinefive

\item \textit{Warmup-1} -- \textbf{close10}
\evallinefive

\item \textit{Warmup-1} -- \textbf{lastDigit}
\evallinefive

\end{enumerate}
\end{exer}

\initialbox

