\section{Basic Style Conventions}

Now that we're developing more complex code, it's important to take a step back and consider what aspects of the code are aesthetically pleasing. This is important for a very practical reason: it's hard to develop, edit, or correct code that is difficult to read. Thus there are common style conventions that should be followed while developing Java code. 

\subsection{Block Indentation}

You've now seen two examples of a \textbf{block} in Java. if statements and method definitions both involved curly braces being opened around a series of statements. Every time you open a new block (that is, you open a curly brace), you should indent all statements within that block. Typically only 2 spaces are used rather than a full tab, since that block may contain another \textit{nested} block, and so on. Indenting helps you visually recognize what blocks each statement belong to. Thus, while reading a Java program, you can quickly skip over a block of code if, for instance, you're not interested in the if statement it belongs to. 

It's a fairly standard rule that you have no more than one statement on a line and always use a full line for a closing curly brace of a block, but there is some debate over whether you should open a curly brace for a block on the same line as an if statement or a method header, or whether you should open it on its own line. For instance:

\begin{code}
int addStuff(int x, int y) {
  return x + y;
}
\end{code}

and also

\begin{code}
int addStuff(int x, int y) 
{
  return x + y;
}
\end{code}

are both considered ``good style'' by many. Mostly it's important that you're consistent with your choice on where to place the opening brace. In this book, we'll consistently use the style of the first example because that's the way the author was raised (and by extension, also now you).

\subsection{Identifier Choice} 

In week \#1 the rules for Java identifiers (the names of variables and methods) were discussed. If you deviate from those rules, you end up with an error (for example, you can't have a variable named 10x). However, by \textit{convention} we try to use variable names that are meaningful and help explain the purpose of a variable. For instance, if you want to average the values of three exam scores, you might call each of the variables holding the exam scores \textit{exam1}, \textit{exam2}, and \textit{exam3} instead of \textit{x}, \textit{y}, and \textit{z}, and you might call the variable holding the average \textit{averageExamScore} or perhaps simply \textit{average}.


\subsection{Inline Comments} 
The code you're now developing is also complex enough that it may not be easy to look at it and understand what it will do (or what it is intended to do). Programmers deal with this problem with a liberal use of inline comments (also called end-of-line comments, since they appear at the end of lines). At any point in a line, you can type two slashes. Everything following those slashes is ignored by the interpreter. When you define methods, it's often helpful to include a comment explaining what a significant block of code, series of statements, or a particularly complex statement is intended to do (in plain english). In fact, it is often helpful to write these comments \textit{first} and then write the code the comments are describing!

If you use several inline comments in a row, try to ensure they're all tabbed up a similar amount to make your code easier to read and aesthetically pleasing. 


\section{while Loops}

While loops are the last piece of the puzzle for a ``real'' programming language (or what computer scientists call a ``Turing Complete'' language). What this means is that, technically, you can now write a program that is capable of doing whatever any other program written in any other language can do. 

While loops are a lot like if statements. In fact, they have a nearly identical syntax. You start with the \textit{while} keyword, then you supply a boolean expression in parentheses, then a block of code (a series of statements surrounded by curly braces). The only difference is that when this block of code completes, program control returns to the beginning of the while loop. Think of it as an if statement that repeats itself indefinitely.

Now you may be concerned by the word ``indefinitely''. You should be -- it's a distinct possibility that a while loop can run ``forever''. However, after each \textit{iteration} of the loop (that is, each time the statements inside the block are executed), the boolean expression at the front of the while loop is evaluated again. If you wrote the loop correctly, that boolean expression should evaluate to false when you're done with the loop. 

For example, the following code will add up the numbers 1 through 10:

\begin{code}
int count = 0;
int total = 0;
while(count <= 10) {
  count = count + 1;
  total = total + count;
}
\end{code}

After the first iteration of the loop, count will increase by 1 and total will increase by 1. Thus, when the boolean expression is evaluated the second time, count will be 1 (and not 0). However, 1 is still less than 10, so the loop will run a second time. After the second iteration, count will increase to 2 and total will increase to 3. Now you might see where this is going. The loop will run 8 more times until count is equal to 11. At that point, count will be greater than 10, so the boolean expression will be false, and the statements inside the block will be skipped. 

\section{More Advanced Conditional Statements}

Last week \textit{else} statements were introduced and we also looked at nesting if statements. This week we will complete our coverage of conditional statements with a few special cases and a new datatype that helps explain the usefulness of the final example. 

\subsection{if/else Chains}

Last week, when a situation called for one of two kinds of action, we could use a if statement with an else statement following it. However, if there are more than two actions we might take, the if statements will get somewhat complicated. To avoid this complexity, there are coding conventions (like those mentioned in the first section) that dictate when you should use curly braces and how much you should indent to avoid very wide lines in your code. This is called using an if/else chain, and is explained with an example:

\begin{exa}
An application is being developed that can order an appropriate snack for you automatically. If the temperature outside is high, it will order an ice cream, and if the temperature is low, it will order a hot chocolate. Assume that there are two methods available that can order either of these foods. The code might look like this:

\begin{code}
if ( temperature > 65 ) {
  orderIceCream();
} else {
  orderHotChocolate();
}
\end{code}

Now consider that the application might also order popcorn if the temperature is more moderate. We would have to use a nested if statement to accomplish this:

\begin{code}
if ( temperature > 80 ) {
  orderIceCream();
} else {
  if (temperature > 50 ) {
    orderPopcorn();
  } else {
    orderHotChocolate();
  }
}
\end{code}

If we wanted to add, for example, 5 more snack options, this would become quite a mess, with many layers of nested if statements inside of else statements! Your code would be so far indented to the right, that it may begin to run over the length of the screen. 

This is avoided by not opening a new curly brace for each if statement that immediately follows an else, and instead putting both on the same line. The body of the else then becomes the if statement that follows the else, and the following if statement is not indented further than the else that it follows. For example, we could re-write the code above this way:

\begin{code}
if ( temperature > 80 ) {
  orderIceCream();
} else if (temperature > 50 ) {
  orderPopcorn();
} else {
    orderHotChocolate();
}
\end{code}

\end{exa}

\subsection{The Character Primitive Data Type}

A character is essentially another type of numeric primitive, but with a special purpose. In order to work with textual data on a computer (such as the letters you're reading right now, which were printed off by a computer), each letter (and also other symbols such as punctuation, numbers, etc) are each given a special code number. 

Characters are a primitive type in Java, and the name of the type is \textbf{char}. Thus, just like \textbf{int} and \textbf{double}, you can create a variable of type \textbf{char}:

\begin{code}
char letterGrade;
\end{code}

There is also a special format for character literals. You can surround a character with single-quotes and Java will interpret that as a literal character value. For instance:

\begin{code}
letterGrade = 'A';
\end{code}

Since Java was designed to work on any available computing platform, naturally it should also support any language as well. Thus, the standard used for the characters in Java include codes not only for the letters used in English, but also all other natural languages (including those with many characters, such as Chinese)! This system is called Unicode:

https://unicode.org/

The actual codes used for each character in Java can be looked up on the UTF-8 chart here:

https://www.unicode.org/charts/


\subsection{switch Statements}

If each of the \textit{if} statements in an if/else chain are simply comparing some expression to one of several literal values, there is a special syntax in Java called a \textbf{switch} statement that can be used instead. Here is an example of such a scenario:

\begin{exa}
A gradebook application needs to determine the minimum grade needed for each letter grade. Letter grades will be represented with capital letter characters. The following method could be used to do the conversion:

\begin{code}
int getMinimumGrade(char letterGrade) {
  if ( letterGrade == 'A' ) {
    return 90;
  } else if ( letterGrade == 'B' ) {
    return 80;
  } else if ( letterGrade == 'C' ) {
    return 70;
  } else if ( letterGrade == 'D' ) {
    return 60;
  } else {
    return 0;
  }
}
\end{code}
\end{exa}

Switch statements begin with the switch keyword and are followed by an expression in parentheses. However, this expression does not need to be a boolean expression. Instead, it should simply produce a value that will be compared to a series of literal values. Following this, a curly brace should be opened for the body of the switch statement. 

The body will contain a series of \textbf{case} statements. Each case statement begins with the \textit{case} keyword, followed by the literal value that the original expression will be compared to, and then a colon. Following that will be a series of statements. Curly braces are not necessary.

Note that after a case statement is complete, the next case statement will be executed! Unless this is intentional, you must explicitly end the switch statement using a \textbf{break} statement. This is simply the keyword \textit{break} followed by a semicolon. Break statements are not needed if a return statement is used, since execution of the enclosing method will end at that point anyway. 

One last concern with switch statements is how to handle the case where the value we get back from the initial expression doesn't match any of the literal values in the cases. We could simply do nothing, or we could include a default action. These is special syntax for the latter. It looks like the other case statements, but uses the keyword \textit{default} followed by a colon instead of the usual beginning of a case statement.

Below is an equivalent implementation of the grade example above using a switch statement instead of an if/else chain:


\begin{code}
int getMinimumGrade(char letterGrade) {
  switch(letterGrade) {
    case 'A':
      return 90;
    case 'B':
      return 80;
    case 'C':
      return 70;
    case 'D':
      return 60;
    default:
      return 0;
  }
} 
\end{code}

\newpage

\section{Before Next Class}

Please answer the following questions before you arrive to class:

\begin{exer}

\begin{itemize}
  
\item Name at least three new keywords you used this week

  \evalline
  
\item What keyword that you used this week is used as a primitive type?

  \evalline
  
\item What kind of statement is a while statement most similar to?

  \evalline
  
\item What kind of statement might be a good alternative to if/else chains?

  \evalline
  
  
\end{itemize}

\end{exer}

Bring your completed pre-lab sheet with you to class for Week \#5. 

