\subsection{Part 1 -- Comparison Operators}

We'll start by working with the most basic comparison operator: the equality operator. Look over the postlab \#1 solution. Note that the equality operator is being used to determine if a is divisible by b (the resulting values are true or false, depending on whether a is divisible by b). 

\begin{exer} Now look over your examples from lab \#1 exercise \#16. Use the equality operator to make sure the expressions you expected to yield equivalent values do so. Write down your new boolean expressions below.

\evallinefour
\end{exer}

\begin{exer}
Create a double variable named x and set it to the value 5.0. Now evaluate two expressions to find $2^x$ and $x^3$, respectively. 

Now develop a boolean expression that is true when $2^x$ is greater than $x^3$. Write it below:

\evalline

It should initially evaluate to false. Increase the value for x and then re-evaluate it. Is it still false? Do this several times until it becomes true, then try to narrow in as closely as you can to the value of x where this expression switches from false to true. What did you find?

\evalline

\end{exer}

\initialbox


\subsection{Part 2 -- Boolean Operators}

\begin{exer}
Determining whether a double value is within a certain range is important in Java since rounding errors prevent us from being able to reliably check for an exact value. It is also a common source of errors for beginning Java programmers! Continuing with the value x from Part \#1, develop an expression which is true when the value of x is between 5 asnd 10 inclusive, and false otherwise. Write it below:

\evalline

\end{exer}

\begin{exer}
Now define two more double variables, y and z. Write an expression which is true when all of x, y, and z are positive, and false otherwise. Make sure to test it out with at least a few values of these variables (positive and negative).

\evalline

Now develop an expression which is true when at least one of these variables is positive. Again, make sure to test it!

\evalline
\end{exer}

DeMorgan's law is an important property of boolean algebra which allows us to distribute the not operator inside of parentheses. Assume you have variables a and b of type boolean. In Java, the law can be stated with the following expression:

\begin{code}
!(a && b) == !a || !b
\end{code}

And also:

\begin{code}
!(a || b) == !a && !b
\end{code}

\begin{exer}
Using this law, re-write your second expression without using the or operator. Make sure it works by testing with various values of x, y, and z!

\evalline
\end{exer}


We typically think of leap years as occurring every 4 years. Generally, these will be the years that are divisible by 4. 


\begin{exer}
Create a variable of type int named year and initially set it to 2018. Develop an expression which is true when year is divisible by 4 and false otherwise. This is a naive solution to developing an expression that is true when year is a leap year. Try it out. 2018 shouldn't be a leap year, but 2020 will be.

\evalline

\end{exer}

I called this a naive expression because the rules for leap years are more complicated than this. If we had a leap year every 4 years, it would over correct and our calendars would fall out of sync. To correct for this, years divisible by 100 are not counted as leap years (for instance, 1900 was not a leap year). However, this over corrects too much in the other direction! To avoid over correcting, we still count years divisible by 400 as leap years (for instance, 2000 was a leap year!)

\begin{exer}
Develop a non-naive expression that is true when year is a leap year and false otherwise:

\evallinethree

\end{exer}

\initialbox


\subsection{Part 3 -- Conditional Expressions}

\begin{exer}
Develop a conditional expression that yields the maximum of x and y.Do not use Math.max(). Try it out with various values of x and y.

\evalline
\end{exer}

\begin{exer}
Develop a conditional expression that yields the absolute value of x. Do not use Math.abs(). Try it out with various values of x.

\evalline
\end{exer}


\initialbox


\subsection{Part 4 -- if Statements}

\begin{exer}
Look back at the boolean expression you used in the second exercise from part 1. Use it within an if statement and add an assignment statement to the body of the if statement that increases x by a small amount. Set x to the smallest value you found. Write your if statement below (include it all on one line):

\evalline

Now repeat this if statement several times by hitting the up arrow. If you adjust x by a very small amount, you should now be able to more quickly narrow in on a more accurate solution. What did you find?

\evalline

\end{exer}

Is it possible to now rewrite our GCD algorithm from postlab \#1 using if statements so that we no longer need to manually manage the logic of our algorithm? Discuss this with your partner and explain your thoughts to the instructor before you receive your last signature. 



\initialbox

Once you've received all the required signatures for this lab, begin work on your postlab assignment. 
