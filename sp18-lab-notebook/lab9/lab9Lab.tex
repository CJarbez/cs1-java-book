\subsection{Part 1 -- Hello World!}

\begin{exer}
Open the notepad++ text editor and create a file called \textbf{Hello.java} in your P: drive folder. Open up the command interpreter as you would to start jshell, but do not start jshell. Enter in the ``Hello World'' example from the prelab in to the text editor and save your work. Then compile and run this file from the command line. If you have trouble with this step, ask for help from the instructor right away. 

Next, run the hello world application from the command line. You should see the greeting printed.
\end{exer}


\begin{exer}
Repeat the steps above for the command-line argument demo from the prelab. When you are finished, try some different command-line arguments. How is the number of command-line arguments determined? Give your answer below:

\evallinetwo
\end{exer}

\initialbox
\subsection{Part 2 -- Command-Line Argument Driven Application}

\begin{exer}
Create a new file called \textbf{Adder.java}. This will be an application that reads in two integer values from the command-line, adds them together, and prints out the result. Note that even if the user enters integers at the command-line, they will be interpreted as strings in your main method. You can convert a string to an integer using the Integer.parseInt() method and providing the string you want to convert as an argument (the return value is an integer). 

Once you have this running, perform some validation on the user input. That is, if the user made a mistake, your program shouldn't crash, but instead should issue the user a warning. Test the application until you're confident that you're aware of all the ways a user could make a mistake and list them below:

\evallinefive

Correct as many as you can in a reasonable amount of time (the instructor will let you know when to move on). When you're finished, email the final solution to your partner if you are the one entering the program. 
\end{exer}


\initialbox
\subsection{Part 3 -- Text-Based Application}

\begin{exer}
Open a new file called \textbf{InteractiveAdder.java} and copy in your code from the last step. Modify your program so that you prompt the user for the values and read them in from the user rather than using the command-line arguments. Test your program thoroughly to ensure it's working properly. 
\end{exer}


\begin{exer}
Open a new file called \textbf{InteractiveCalculator.java} and copy in your code from the last step. Modify your program so that the user is first presented with a menu that has three options:

\begin{enumerate}
\item Add two numbers together
\item Multiply two numbers together
\item Quit
\end{enumerate}

The user should then be prompted for their choice of action. If the user selects an invalid option, they should be warned and given the menu again. If they select the first option, the program should follow the same logic as the previous example, but after adding the numbers and printing the result, the user should be returned to the menu. Option 2 should work similarly. Option 3 should be the only way to end the program.
\end{exer}


\initialbox
\subsection{Part 4 -- More Array Practice}
Complete the following problems on codingbat:

\begin{itemize}
\item sum28 - http://codingbat.com/prob/p199612
\item matchUp - http://codingbat.com/prob/p136254
\end{itemize}

\initialbox
