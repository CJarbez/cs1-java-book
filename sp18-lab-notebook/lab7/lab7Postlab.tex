For your postlab \#7 assignment, you will write a void method using the jshell REPL. Write a method called \textbf{printCharactersInString}. This method should have no return value (void) and should have one parameter of type String named \textit{str}. Your method should print each character in the string in order and one character per line. The trick is that your method should not repeat any characters more than once. For example, calling:

\begin{code}
printCharactersInString("hello there!")
\end{code}

Should print the following:

\begin{verbatim}
h
e
l
o
 
t
r
!
\end{verbatim}

There are multiple ways to get this working correctly. You could use a nested loop, or you could use some of the instance methods in the string class. It is up to you how you implement this method so long as it works!

Make sure to include a descriptive javadoc comment at the top of your method definition, and end-of-line comments at appropriate places inside your method. 

When you are finished, copy and paste your method definition out of the editor in jshell and in to the submission box in Moodle. 
