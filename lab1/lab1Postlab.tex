
For your postlab assignment, you will need to follow the euclidean algorithm with 
the REPL to find the greatest common factor of several sets of numbers. Given 
integers a and b, the algorithm could be described as follows:

\begin{enumerate}
\item if a is less than b, swap their values
\item if a is divisible by b, then b is the GCD (you're done). 
\item determine the remainder of a and b and set it to a variable r.
\item set a to the value in b.
\item set b to the value in r.
\item go back to step 1.
\end{enumerate}

Your task will be to compute the GCD's for the following pairs of values:
\begin{itemize}
\item 15 and 25
\item 84 and 140
\end{itemize}

Use the JShell REPL to follow the algorithm and make sure all the steps are visisble in the output from your REPL, including the correct answers. 

When you are finished, enter the following command in to the REPL:

\begin{verbatim}
/save postlab1.txt
\end{verbatim}

Check to make sure this file contains all of your work from the REPL, then 
turn it in on the postlab \#1 assignment. 