Evaluating expressions and statements in the REPL is fine and good, but in Java to truly develop an \textbf{application} that can be distributed to other users, there are a few more concepts that need to be addressed. We'll take a look at these concepts in this lab.

\section{Basic Class Definitions}

Every object type has a class definition associated with it, but classes in Java are also used for other special purposes, such as creating a collection of static methods (for example, the Math class) or for creating a Java application. 

\subsection{Class and Static Method Definition Syntax}

A class definition begins with the keyword \textbf{class}, followed by the class name, and then a pair of curly braces. For example, here is a very simple class definition:

\begin{code}
class Lab9 {
}
\end{code}

A Java file must contain one class designated as public using the \textbf{public} keyword before the class definition. It's also important to note that the filename must match the name of the public class, with \textit{.java} at the end (similar to how an HTML file has \textit{.html} at the end). For example, if we were to produce a Java file called \textit{Lab9.java}, it could have the following as its contents:

\begin{code}
public class Lab9 {
}
\end{code}

Inside of these curly braces is the class body in which its \textbf{members} are defined. Class members include method definitions, so we can include any the method definitions we've been developing. If we make these methods static by including the \textit{static} keyword before the method definition, we can use them like any other static method (such as Math.max). For instance, consider the following modification of the Lab9 class definition:


\begin{code}
public class Lab9 {
  static void sayHello() {
    System.out.println("Hello!");
  }
}
\end{code}

Any other Java code we write, even in another file, could access this method as follows:

\begin{code}
Lab9.sayHello();
\end{code}



\subsection{Static Fields}

Methods aren't the only members we can add to a class. Variables can also be members, in which case they are called \textbf{fields}. For example, we can create a variable called \textit{greeting} to hold the message we use in the sayHello() method. This allows us to more easily maintain consistency with other methods that greet the user in other ways. \textit{greeting} must also be declared static to be used in Lab9's static methods: 


\begin{code}
public class Lab9 {
  static String greeting = "Hello!";

  static void sayHello() {
    System.out.println(greeting);
  }

  static void sayHelloEmphatically() {
    System.out.println(greeting + "!!!!!!!!");
  }
}
\end{code}

\subsection{Scope of Fields} 

Note that the variable \textit{greeting} in the previous example is accessible in both of the static methods in the class Lab9. That is because the \textbf{scope} of the variable (that is, the part of the code where the variable is accessible) is the entire class. For the most part, the scope of a variable is limited to all parts of the code within the curly braces it's defined in (or just before, such as with method parameters and for loops). 

\subsection{Constants}

Because \textit{greeting} is a member of the class, just like the sayHello() method, it can be accessed from outside of the class as well. That means that the greeting could be modified from outside the class, such as in the following example:

\begin{code}
Lab9.greeting = "Hey Jerk";
\end{code}

If a field is never meant to be altered, it should be made in to a \textbf{constant} by using the \textbf{final} keyword in its definition. Here is the updated definition of the Lab9 class that prevents any changes to the greeting field:


\begin{code}
public class Lab9 {
  static final String greeting = "Hello!";

  static void sayHello() {
    System.out.println(greeting);
  }

  static void sayHelloEmphatically() {
    System.out.println(greeting + "!!!!!!!!");
  }
}
\end{code}

\section{Java Applications}
The last section explained how Java classes can be used to hold collections of static methods. They are also used as the entry points in to stand-alone Java applications. 

\subsection{Main Methods}
To run a java application, you must provide the Java runtime environment (JRE) with the name of a class representing a Java application. The JRE will then search for one particular static method called \textbf{main} and execute it. This is called the \textbf{entry point} to the application. 

The JRE is very picky about how main should be defined. Firstly, the name must be exact -- \textit{main} must not be capitalized or otherwise misspelled. Next, the method (like the class definition) must be declared to be public. It also must be a static method, so both of these keywords need to preceed the rest of the method definition (typically public comes first, but the order is more of a style issue). Next, the return type of the method must be void. Finally, the method must have one parameter. The parameter can be named anything you want, though by convention it is given the name \textit{args}. What is important is that the type of this parameter is an array of strings:  String[].  

Putting this all together, the following serves as an example of a complete Java application:

\begin{code}
public class Hello {
  public static void main(String[] args) {
    System.out.println("Hello World!");
  }
}
\end{code}

This is the classic ``hello world'' program that is often given on the first day of a programming class. In many languages, ``Hello World'' is simply a one-line program, making this a convienent place to start in an introductory programming class. However, Java was designed with more complex software development in mind, and a more complex ``Hello World'' example is one of the unintended side effects of that choice (pun intended). By starting with tools like jshell and codingbat, you were able to learn each of the concepts present in this code before using them in an application (with the exception of the \textit{public} keyword, to some extent). 

Note that this program would have to be defined in a file called \textbf{Hello.java}. 

\subsection{Compiling and Running Java Programs}

If you develop a java file such as \textbf{Hello.java}, you must first compile it before you run it. Recall that typing \textit{jshell} at the command prompt started up the jshell REPL. Similarly, you can compile this Java program at the command prompt by typing ``javac Hello.java''. If the command was successful, you should see a file in your directory called ``Hello.class''. If it was unsuccessful, you will recieve an error message. Generally, you can compile any other Java file by including its file name in the place of ``Hello.java''. 

Once your program is compiled successfully, you can run it by typing the command ``Java Hello''. This starts up the JRE and tells it to look for a main method in the class \textbf{Hello}. If you did not properly define your main method, you may receive an error message at this point. Otherwise, you will see the welcome message printed in the console.   


\subsection{Command-Line Arguments}

Though you should understand what the type \textbf{String[]} means at this point (an array of string values), it may not be clear why a parameter with this type must be provided to the main method. The purpose of this parameter is to hold any \textbf{command-line arguments} that may be provided to your application when it is started at the command prompt. This would allow you to write an application similar to \textit{javac} itself! 

Consider the following example program:

\begin{code}
public class CommandLineDemo {
  public static void main(String[] args) {
    for(int i=0; i<args.length; i++) {
      System.out.println(args[i]);
    }
  }
}
\end{code}

This application prints out each of the command line arguments given to it when it is executed. These arguments are specified at the end of the \textit{java} command that starts the JRE. 

For example, if you were to compile and run this program with the following command:

\begin{verbatim}
java CommandLineDemo here are some arguments
\end{verbatim}

...the following would be printed to the screen:

\begin{verbatim}
here
are
some
arguments
\end{verbatim}



\section{Console I/O}

A \textbf{text-based application} (or command-line application) is one type of application you might develop and a simple one to start with. This type of application uses the command prompt to interact with the user. This was a very popular style of application on older computer systems in the 90's and earlier but has remained relevant and has perhaps even gained popularity more recently. For example, jshell itself is a text-based application.

\subsection{Writing to Standard Output with the System Console}

A text-based application writes to standard output to communicate something to the user. You've already seen how to do this by using \textbf{System.out.println}. What you haven't seen yet is how to receive a response from the user while your program is running. This chapter will introduce the system console which is capable of doing both. Note that the console can't be used in jshell (if you try, you will get an error), but it can be used in a stand-alone text-based application run from the command-line. 

The system console can be retrieved using the console() static method of the System class. This method returns a \textbf{Console} object with several instance methods we can use for input or output. One such instance method, the \textbf{printf} method, works similarly to the System.out.print method we used earlier. Here is ``Hello World'' using this method from the console:

\begin{code}
public class Hello {
  public static void main(String[] args) {
    System.console().printf("Hello World!");
  }
}
\end{code}

printf allows us to more easily print complex messages without having to concatenate several expressions together. This is done using a formatted string -- a string containing special character sequences that are a stand-in for other values provided as additional arguments to printf. 

\begin{exa}
``\%d'' is the character sequence that is used for an integer argument, and the following will print out the contents of the \textit{age} variable without using string concatentation:

\begin{code}
int age = 28;
System.console().printf("I am \%d years old.", age);
\end{code}
\end{exa}

To see what other types of character sequences you can use, check the documentation for printf method of the java.io.Console class. 

\subsection{Reading from Standard Input Using the System Console}

The Console class also has several useful methods for reading input in from the user. The most essential is \textbf{readLine}. Calling this method with no arguments will have it simply return the next line of input from the user as a single string. The program will pause and allow the user to type until they press the enter / return key, at which point the line of text that the user entered will be returned as a string. 

Since the application will pause for the user's response when this method is called, the user should be presented with a \textbf{prompt} to indicate that they are expected to enter some input. Such a prompt can optionally be provided as an argument to the readLine method. 


\begin{exa}
The following application will quiz the user on some basic Java vocabulary. It will start by prompting the user with a question and then indicate to the user whether they entered the correct answer or not.


\begin{code}
public class JavaVocabQuiz {
  public static void main(String[] args) {
    String question = "What do you call a variable that holds an argument value for a method?";
    String answer = System.console().readLine(question + " ");
    if(answer.equals("parameter")) {
      System.console().printf("Correct!\n");
    } else {
      System.console().printf("Wrong!\n");
    }
  }
}
\end{code}

\end{exa}


\newpage

\section{Before Next Class}

Please answer the following questions before you arrive to class:

\begin{exer}

\begin{itemize}

\item What is the name of a method that must be in every Java application? 

  \evalline
  
\item What keywords must come before the name of the method in a definition of the method above? 

  \evalline
  
\item In a text-based application, how can you read a line of input from the user?

  \evalline
  
\end{itemize}

\end{exer}

Bring your completed pre-lab sheet with you to class for Week \#7. Also submit your answers online on the Moodle prelab assignment.  

