\subsection{Part 1 -- For-Each Loops}

\begin{exer}
Take a look through some of the codingbat exercises that you've solved with for loops. Find one that can be solved with a for-each loop and update it in codingbat. For your records, record the name of the codingbat problem you updated (include the p\# in the URL):

\evalline
\end{exer}

\begin{exer}
Now look through your codingbat work again and find a problem you solved with a for loop that \textbf{cannot} be solved with a for-each loop. Again record the name of that problem below:

\evalline

Be prepared to justify your choice to the instructor when they check your work.
\end{exer}

\initialbox

\subsection{Part 2 -- Varargs and Overloading}

\begin{exer}
In jshell, create a new class called Lab10 with no definitions inside of it. 
After it is accepted by the shell, open it up with the editor by typing /edit Lab10.

Now add a static method named \textbf{average} that takes a variable number of textit{double} arguments. This method should return the average of all of the arguments passed in and return the average as a double. Once your definition is correct, you should be able to evaluate the following statement from jshell:

\begin{code}
Lab10.average(5.0, 9.0, 17.0, 9.0)
\end{code}

and it should yield the value 10.0. Also test it out with some other numbers to ensure it's working properly
\end{exer}

\begin{exer}
Now modify the Lab10 class again and add an overloaded definition of textbf{average} that will ``average'' a variable number of boolean arguments. This method should interpret a \textit{true} value as a 1 and a \textit{false} value as a 0. When you're finished, the previous version of average should still work (that is, you should still get the same result from typing in the expression from the last exercise), but you should also get the value 0.25 from evaluating the following expression:

\begin{code}
Lab10.average(false, true, false, false)
\end{code}
\end{exer}

\initialbox
\subsection{Part 3 -- Finish Postlab 9}

Postlab #9 is essentially a more complex ``hello world'' with some basic user I/O. It is a nice lead-in to project \#3, so it is recommended that you complete it before beginning project \#3.

\initialbox
\subsection{Part 4 -- Continue with Project 3 and Postlab 10}
While you're in class, work through project 3 and talk to your instructor about any sticking points in your progress. In particular, make sure you can make the connection between your work on postlab \#8, lab \#9, and postlab \#9 with this project. You are allowed to re-use your code from these assignments on your project. If you finish early, move on to postlab \#10.

Make sure to complete Postlab #10 before the final exam next week. This will
be the best practice for the programming questions on the exam. 

\initialbox
