\subsection{Part 1 -- Basic Array Manipulation}

\begin{eval}
\begin{sevalenum}

\item 
Execute the following statement in the REPL:

\begin{code}
String[] colors = new String[2];
\end{code}

Predict what the following expression will evaluate to, then check and confirm your guess:

\begin{verbatim}colors[0]\end{verbatim}

\evallinetwo

\item 

Populate this slot of the array with the value \textbf{"blue"}. Re-evaluate the expression above to confirm you were successful, then write down the statement you used to do this:

\evalline

\item Copy the value you placed in slot 0 in to slot 1. Do not use another literal string. Again, write down the statement you used to do this:

\evalline

\item Now predict the value for the following expression and then confirm / correct your guess:

\begin{code}
colors[1].equals(colors[0])
\end{code}

\evallinetwo

\end{sevalenum}
\end{eval}




\begin{eval}
Typing in each of the following segments of code will result in an error of some sort. Try to predict what it will be, then evalute the code and briefly summarize the error output from the REPL, then  explain what the real problem is on the line below.

\begin{sevalenum}

\item 

\begin{code}
new int[3] {1, 2, 3}
\end{code}
\evallinethree


\item 

\begin{code}
int[] arr = new int[] {1, 2, 3};
arr[arr.length]
\end{code}
\evallinethree



\item 

\begin{code}
String[] foods = new String[2];
foods[0].equals(foods[1])
\end{code}
\evallinethree

\end{sevalenum}
\end{eval}


\initialbox

\subsection{Part 2 -- Working with Object References}


\begin{eval}
Enter the statements for each of the following examples and predict what the final expression will evaluate to. Then evaluate the expression and confirm / correct your guess on the second line. 

\begin{sevalenum}




\item 
Type in the following statements:
\begin{code}
import java.util.Point;
Point[] locations = new Point[3];
locations[0] = new Point(3, 7);
locations[1] = new Point(1, 2);
locations[2] = new Point(9, 3);
\end{code}

Predict / Correct the value for the following expression:

\begin{code}
locations[1].getX()
\end{code}
\evallinetwo



\item

Type in the following statement:

\begin{code}
Point[] locations2 = locations;
\end{code}

Predict / Correct the value for the following expression:

\begin{code}
locations2[0].getY()
\end{code}

\evallinetwo

\item 

Type in the following statement:

\begin{code}
locations2[0] = new Point(8, 5);
\end{code}

Predict / Correct the value for the following expressions (seperate each value by commas on both lines):

\begin{code}
locations[0].getX()
locations[0].getY()
locations2[0].getX()
locations2[0].getY()
\end{code}

\evallinetwo

\item 

Type in the following statement:

\begin{code}
locations2 = new Point[3];
locations2[0] = locations[0];
locations2[1] = locations[1];
locations2[2] = locations[2];
\end{code}

Predict / Correct the value for the following expressions (seperate each value by commas on both lines):

\begin{code}
locations[0].getX()
locations[0].getY()
locations2[0].getX()
locations2[0].getY()
\end{code}

\evallinetwo

\item 

Type in the following statement:

\begin{code}
locations2[0] = new Point(12, 0);
\end{code}

Predict / Correct the value for the following expressions (seperate each value by commas on both lines):

\begin{code}
locations[0].getX()
locations[0].getY()
locations2[0].getX()
locations2[0].getY()
\end{code}



\item 

Type in the following statement:

\begin{code}
locations[1].setLocation(20, 25);
\end{code}

Predict / Correct the value for the following expressions (seperate each value by commas on both lines):

\begin{code}
locations[1].getX()
locations2[1].getX()
\end{code}

\evallinetwo


\end{sevalenum}
\end{eval}

\initialbox


\subsection{Part 3 -- Simple Array Methods}

\begin{exer}
Complete the following three Array-1 problems on codingbat:
\begin{enumerate}
\item sum3 - http://codingbat.com/prob/p175763
\item makePi - http://codingbat.com/prob/p167011
\item rotateLeft3 - http://codingbat.com/prob/p185139

\end{enumerate}
\end{exer}

\initialbox
\subsection{Part 4 -- Iterating Through Arrays}

\begin{exer}
Complete the following Array-2 problem on codingbat:

countEvens - http://codingbat.com/prob/p162010

\end{exer}

\initialbox
