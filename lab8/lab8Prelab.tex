\section{Arrays}

Strings are a collection of characters, but we haven't yet seen a way to handle a collection of other types of values. Arrays allow Java programmers to create collections of values of an arbitrary type. 

\subsection{The Array Type}

Values in an array can have any type, but they must all have the \textit{same} type. Arrays themselves are a type of data, but are \textit{parameterized} by the type of data they hold. Thus an array type has two parts: the type of data held by the array, and a pair of square brackets indicating an array type. For example, they type ``array of integers'' would be written:
\begin{code}
int[] 
\end{code}

A variable named \textit{arr} that could hold an array of integers is defined as follows:

\begin{code}
int[] arr;
\end{code}

\subsection{Creating Array Values}

\subsection{Operations on Arrays}

\section{Object Parameter Passing}
\section{For-Each Loops}

\section{Before Next Class}

Please answer the following questions before you arrive to class:

\begin{exer}

\begin{itemize}

\item ***

  \evalline
  
\item ***

  \evalline
  
\end{itemize}

\end{exer}

Bring your completed pre-lab sheet with you to class for Week \#7. Also submit your answers online on the Moodle prelab assignment.  

