\section{Booleans}

This section will introduce a significant new primitive type: \begin{textbf}booleans\end{textbf}.

\subsection{The Boolean Primitive Type}

The boolean type is actually the most simple type because they have only two possible values: \begin{textit}true\end{textit} and \begin{textit}false\end{textit}. Both \begin{textit}true\end{textit} and \begin{textit}false\end{textit} are also literal boolean values, and \begin{textit}boolean\end{textit} is the name of the type. For instance, you can declare and initialize a boolean variable as follows:

\begin{code}
boolean mybool = true;
\end{code}


\subsection{Comparison Operators}
The boolean type is useful for answering ``yes'' or ``no'' quesions, such as ``is 2 equal to 1 + 1?''. We can ask these types of questions with comparison operators. 

The comparison operators are binary operators, just like the arithmetic operators you learned last week. However, they operate over values of any type, and the evaluate to a boolean value. For instance, == is the comparison operator. It determines whether two values are the same or not. So, to answer the question we posed in the last paragraph, we could use an expression like this:

\begin{code}

(1 + 1) == 2

\end{code}

If you type this in to the REPL, it will evaluate to true. Note that the parentheses are not necessary, either. + will be evaluated before any comparison operators will be.

There are several other comparison operators that will be useful to us, summarized below:

\begin{itemize}
\item !=  Not equal to
\item $>$ greater than
\item $<$ less than
\item $>$$=$ greater than or equal to
\item $<$$=$ less than or equal to
\end{itemize}

For example, the expression $4 < 3 + 3$ would evalute to \begin{textit}true\end{textit}. 

\subsection{Boolean Operators}

There are also operators that work strictly over boolean values, just as the arithmetic operators we used last week worked strictly over numeric values. These are called \begin{textbf}boolean operators\end{textbf}. They operate over one or two boolean values and evaluate to another boolean value.  These operations are quite simple, but form the basis for what is known as \begin{textbf}boolean algebra\end{textbf}. 

The first operator we'll consider is the \begin{textit}and\end{textit} operator: \&\&. This is a boolean operator that evaluates to true only when both the left and right operands also evaluate to true. For instance:

\begin{code}

true && false

\end{code}

...evaluates to false, since the right-hand operand is false. However:

\begin{code}

true && true

\end{code}

...evalutes to true, since both operands are true.

The next operator we'll consider is the \begin{textit}or\end{textit} operator: $||$ It is similar to the and operator, but it evaluates to true when either of the operands are true. Put another way, this operator will only evaluate to false when both of its operands evaluate to false. For example, if you replaced \&\& with $||$ in both of the last two expressions, they would both evaluate to true.

The last boolean operator we'll consider is unique in that it's a unary operator. Recall that this means it has only one operand. It is called the not operator: !   This operator simply reverses the boolean value of its operand. For example:

\begin{code}

!false

\end{code}

...evaluates to true. Note that all of these operators can be combined and nested in subexpressions. For instance, the following expression evaluates to true:

\begin{code}

!((false && false) || false)

\end{code}

These operators also have an oerder of precedence: !, $||$, \&\&

\section{Making Decisions}

Booleans are important because you can use them to make decisions in your programs. Note that several of the algorithms we've worked with asked us to do one of two things depending on some condition. Boolean expressions allow us to specify the condition, and conditional expressions and if statements allow us to 
make conditional actions. 

\subsection{if Statements}

An if statement is a new kind of statement that will only execute when a certain boolean expression evaluates to true. These statements start with the keyword \begin{textbf}if\end{textbf}, then a boolean expression (surrounded in parentheses -- the parentheses are necessary), then the statement we want to conditionally execute. For instance, consider the following statements:

\begin{code}
int x = 10;
int y = 0;

if (y > x) x = 30;

if (x > y) y = 20;

\end{code}

The first if statement does nothing. y is not greater than x at this point, so the assignment statement that follows it (x = 30;) does not execute. The second if statement will set the variable y to the value 20. Since x is greater than y, the statement (y = 20;) will be executed. 

More than one conditional statement can be included in an if statement. In fact, you can include as many as you like by opening up a \begin{textbf}block\end{textbf} of code using curly braces surrounding each of the conditional statements:

\begin{code}

if (x > y) {
  x = 30;
  y = 40;
  z = 50;
}

\end{code}

\subsection{Conditional Expressions}
You can also create expressions that evaluate conditionally. These expressions use two operators in conjunction: a question mark (?) and a colon (:). Conditional expressions consist of 3 parts: a boolean expression, followed by a question mark, followed by an expression that wll be evaluated if the initial  boolean expression is true, followed by a colon, followed by an expression that will be evaluated if the initial boolean expression is false.

For example, consider the following expression:

\begin{code}

5 > 10 ? 30 : 40

\end{code}

This expression evaluates to the value 40 since the initial boolean expression evaluates to false, the expression following the colon is evaluated. The middle expression is ignored in this case. If we changed the comparison operator to $<$, the conditional expression would have evaluated to 30. 

Here's another example that would evaluate to the maximum of x and y:

\begin{code}

x > y ? x : y

\end{code}

\newpage

\section{Preparing for Class}

Please answer the following questions before you arrive to class:

\begin{exer}

\begin{itemize}
  
\item Give an example of one of the two boolean values

  \evalline

\item What operator do you use to determine if two values are the same? (note that this is different from the assignment operator!)

  \evalline
  
\item What is the keyword used in an if statement? 

  \evalline
  
\item What special operators are used in conditional statements?

  \evalline
  
  
  
\end{itemize}

\end{exer}

Please bring your completed pre-lab sheet with you to class.
